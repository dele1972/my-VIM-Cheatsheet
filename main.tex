% If you're new to LaTeX, the wikibook is a great place to start:
% http://en.wikibooks.org/wiki/LaTeX

\providecommand{\myVersionnumber}
    {0.2}

\providecommand{\myTitle}
    {my VIM Cheatsheet}

\providecommand{\myAuthor}
    {Dennis Lederich}

% point url to the GitHub project
\providecommand{\myGithublink}
    {https://github.com/dele1972/} % provides '\myTitle', '\myVersionnumber', '\myGithublink', ...

\documentclass[10pt]{article} % 10pt font size

% packages
    \usepackage[utf8]{inputenc} % Required for inputting international characters
    \usepackage[T1]{fontenc} % Output font encoding for international characters
    \usepackage{multicol,multirow} % Required for dynamically ordered column-layout
    \usepackage[margin=1cm,landscape,a4paper]{geometry} % Page margins and orientation
    \usepackage[dvipsnames]{xcolor} % Required for color customization
                                    % (https://en.wikibooks.org/wiki/LaTeX/Colors)
    \usepackage[ % This block contains information used to annotate the PDF
            colorlinks=true,
            urlcolor=NavyBlue,
            pdftitle={\myTitle}, 
            pdfauthor={\myAuthor}, 
            pdfsubject={}, 
            pdfkeywords={}
        ]{hyperref}

% add biblatex for citation
    \usepackage[backend=biber,style=numeric,]{biblatex}
    \addbibresource{cites.bib}

\pagestyle{empty}

\setcounter{secnumdepth}{0}
\setlength{\parindent}{0pt}
\setlength{\parskip}{0pt plus 0.5ex}
\setlength{\unitlength}{1mm} % Set the length that numerical units are measured in
\setlength{\parindent}{0pt} % Stop paragraph indentation

% define some custom layout by 'overriding' commands
    \makeatletter   % refer '@' to Categorycode 11,
                    % now it could be used as naming part of command sequences
    % define 'section' layout:
    \renewcommand{\section}{\@startsection{section}% name
                                    {1}% level
                                    {0mm}% indent
                                    {2ex plus -.5ex minus -.2ex}% beforeskip
                                    {1ex plus .2ex}% afterskip
                                    {\color{WildStrawberry}\normalfont\Large\bfseries}% style
                                    }
    
    % define 'subsection' layout:
    \renewcommand{\subsection}{\@startsection{subsection}{2}{0mm}%
                                    {-2explus -.5ex minus -.2ex}%
                                    {0.5ex plus .2ex}%
                                    {\normalfont\normalsize\bfseries}}
    
    % define 'subsubsection' layout:
    \renewcommand{\subsubsection}{\@startsection{subsubsection}{3}{0mm}%
                                    {-2ex plus -.5ex minus -.2ex}%
                                    {1ex plus .2ex}%
                                    {\normalfont\small\bfseries}}
    \makeatother    % refer '@' back to Categorycode 12,
                    % now it would be used as Textcharacter

% define custom commands for an alternate layout of a command list
    \renewcommand{\dots}{\ \dotfill{}\ } % Fills in the right amount of dots
    \newcommand{\command}[2]{#1~\dotfill{}~#2\\} % Custom command for adding a dotted shorcut
    \newcommand{\commandbl}[2]{#1~\hfill~#2\\} % Custom command for adding a shorcut with blanks

% Custom command for alternate subsection titles
    \newcommand{\sectiontitle}[1]{\paragraph{#1} \ \\  \medskip\hrule\medskip} 

%%%%%%%%%%%%%%%%%%%%%%%%%%%%%%%%%%%%%%%%%%%%%%%%%%%%%%%%%%%%%%%%%%%%%%%%%%%%%%%%%%%%%%%%%%%%%%%%

\title{\mytitle}
\begin{document}
\raggedright
\footnotesize

% ----------------------------------------------------------------------- Title
\begin{center}
    \Large{\textbf{\myTitle}} \\
    \Large{\normalsize Version \myVersionnumber} \\
\end{center}
\medskip

% ----------------------------------------------------------------------- start multicol
\begin{multicols}{3}
\setlength{\premulticols}{1pt}
\setlength{\postmulticols}{1pt}
\setlength{\multicolsep}{1pt}
\setlength{\columnsep}{2pt}

These shortcuts and descriptions are almost get from the awesome guide: \textit{"Onramp to Vim"}\cite{onramp}. Another ressource to start with Vim is the online Version of the vimtutor\cite{onlinetut}.

% ----------------------------------------------------------------------- motions
% https://thoughtbot.com/upcase/videos/onramp-to-vim-motions-and-moving
\section{Motions (nouns)}
Motions are in the sense of movement.\\
See Vim's help page for motions for a full listing: \texttt{:h motion}

\medskip
\subsection{Moving within a line}
\command{\texttt{h}, \texttt{l}}{move left/right by character}
\command{\texttt{w}}{move forward one (\textbf{w})ord}
\command{\texttt{b}}{move to (\textbf{b})eginning of a word}
\command{\texttt{e}}{move forward to the (\textbf{e})nd of a word}

\medskip
\subsection{Jumping within a line}
\command{\texttt{f<char>}}{(\textbf{f})ind a character forward in a line and move to it}
\command{\texttt{t<char>}}{find a character forward in a line and move un(\textbf{t})il it}
\commandbl{}{(one character before)}
\command{\texttt{F<char>}}{(\textbf{f})ind a character backward in a line and move to it}
\command{\texttt{T<char>}}{find a character backward in a line and move un(\textbf{t})il it}
\command{\texttt{;}}{repeat last \texttt{f}, \texttt{t}, \texttt{F}, or \texttt{T} command}
\command{\texttt{,}}{repeat last \texttt{f}, \texttt{t}, \texttt{F}, or \texttt{T} command, but in opposite direction}

\medskip
\subsection{Moving between lines}
\command{\texttt{j}, \texttt{k}}{move up/down one line}
\command{\texttt{H}, \texttt{M}, \texttt{L}}{move (\textbf{H})igh, (\textbf{M})iddle, or (\textbf{L})ow within the viewport}
\command{\texttt{/}}{search}
\command{\texttt{n}}{repeat last search - (\textbf{n})ext}
\command{\texttt{N}}{repeat last search in opposite direction}
\command{\texttt{<ctrl>u}, \texttt{<ctrl>d}}{moves (\textbf{u})p or (\textbf{d})own half of a page}
\command{\texttt{<NN>G}}{(\textbf{G})o to line number \textit{NN}}

%\columnbreak


% ----------------------------------------------------------------------- text-objects
% ToDo: here i am - https://thoughtbot.com/upcase/videos/onramp-to-vim-command-language
\section{Text Objects (nouns)}
text objects allow you to run the command from anywhere inside the text object\\
For a full listing, see \texttt{:h text-objects}


% ----------------------------------------------------------------------- commands
\section{Commands (verbs)}
After pressing the key-mapping for a given command Vim will wait for you to identify the text you want the command to operate on. The simplest commands are made by simply repeating the operator a second time to act on the current line.\\
For example, where \texttt{d} is the operator for "delete", \texttt{dd} will delete the whole line. Each of \texttt{yy}, \texttt{cc}, \texttt{>\@>}, \texttt{==} behave similarly.

\medskip
\command{\texttt{d}}{(\textbf{d})elete}
\command{\texttt{c}}{(\textbf{c})hange}
\command{\texttt{y}}{(\textbf{y})ank (copy)}
\command{\texttt{v}}{(\textbf{v})isually select}
\command{\texttt{>}, \texttt{<}}{indent, dedent}
\command{\texttt{=}}{reformat (reindent, break long lines, etc.}

\medskip
\subsection{combination: Command + Motion}
The following table shows some of the many variations of the delete operation you can build by combining the \texttt{d} operator mapping with a motion:

\medskip
\command{\texttt{dw}}{(\textbf{d})elete to the next (\textbf{w})ord}
\command{\texttt{de}}{(\textbf{d})elete to to the (\textbf{e})nd of the \underline{current} word}
\command{\texttt{d2e}}{(\textbf{d})elete to to the (\textbf{e})nd of the \underline{next} word}
\command{\texttt{dj}}{(\textbf{d})elete down a line (current and one below)}
\command{\texttt{dt)}}{(\textbf{d})elete forward un(\textbf{t})il closing paranthesis}
\command{\texttt{d/world}}{(\textbf{d})elete up until the first search match for \textit{world}}


%%%%%%%%%%%%%%%%%%%%%%%%%%%%%%%%%%%%%%%%%%%%%%%%%%%%%%%%%%%%%%%%%%%%%%%%%%%%%%%%%%%%%%%%%%%%%%%%

% ----------------------------------------------------------------------- Resources
\printbibliography


% ----------------------------------------------------------------------- Endnote
\hrule\medskip
\today ~ \myAuthor, \href{\myGithublink}{\myGithublink}

\end{multicols}
\end{document}