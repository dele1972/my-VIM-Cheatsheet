% If you're new to LaTeX, the wikibook is a great place to start:
% http://en.wikibooks.org/wiki/LaTeX

\providecommand{\myVersionnumber}
    {0.3}

\providecommand{\myTitle}
    {my VIM Cheatsheet}

\providecommand{\myAuthor}
    {Dennis Lederich}

% point url to the GitHub project
\providecommand{\myGithublink}
    {https://github.com/dele1972/} % provides '\myTitle', '\myVersionnumber', '\myGithublink', ...

\documentclass[10pt]{article} % 10pt font size

% packages
    \usepackage[utf8]{inputenc} % Required for inputting international characters
    \usepackage[T1]{fontenc} % Output font encoding for international characters
    \usepackage{multicol,multirow} % Required for dynamically ordered column-layout
    \usepackage[margin=1cm,landscape,a4paper]{geometry} % Page margins and orientation
    \usepackage[dvipsnames]{xcolor} % Required for color customization
                                    % (https://en.wikibooks.org/wiki/LaTeX/Colors)
    \usepackage[ % This block contains information used to annotate the PDF
            colorlinks=true,
            urlcolor=NavyBlue,
            pdftitle={\myTitle}, 
            pdfauthor={\myAuthor}, 
            pdfsubject={}, 
            pdfkeywords={}
        ]{hyperref}
    \usepackage{luacolor} % Required to use the lua-ul \highLight command 
    \usepackage{lua-ul} 

% add biblatex for citation
    \usepackage[backend=biber,style=numeric,]{biblatex}
    \addbibresource{cites.bib}

\pagestyle{empty}

\setcounter{secnumdepth}{0}
\setlength{\parindent}{0pt}
\setlength{\parskip}{0pt plus 0.5ex}
\setlength{\unitlength}{1mm} % Set the length that numerical units are measured in
\setlength{\parindent}{0pt} % Stop paragraph indentation

% define some custom layout by 'overriding' commands
    \makeatletter   % refer '@' to Categorycode 11,
                    % now it could be used as naming part of command sequences
    % define 'section' layout:
    \renewcommand{\section}{\@startsection{section}% name
                                    {1}% level
                                    {0mm}% indent
                                    {2ex plus -.5ex minus -.2ex}% beforeskip
                                    {1ex plus .2ex}% afterskip
                                    {\color{WildStrawberry}\normalfont\Large\bfseries}% style
                                    }
    
    % define 'subsection' layout:
    \renewcommand{\subsection}{\@startsection{subsection}{2}{0mm}%
                                    {-2explus -.5ex minus -.2ex}%
                                    {0.5ex plus .2ex}%
                                    {\normalfont\normalsize\bfseries}}
    
    % define 'subsubsection' layout:
    \renewcommand{\subsubsection}{\@startsection{subsubsection}{3}{0mm}%
                                    {-2ex plus -.5ex minus -.2ex}%
                                    {1ex plus .2ex}%
                                    {\normalfont\small\bfseries}}
    \makeatother    % refer '@' back to Categorycode 12,
                    % now it would be used as Textcharacter

% define custom commands for an alternate layout of a command list
    \renewcommand{\dots}{\ \dotfill{}\ } % Fills in the right amount of dots
    \newcommand{\command}[2]{#1~\dotfill{}~#2\\} % Custom command for adding a dotted shorcut
    \newcommand{\commandbl}[2]{#1~\hfill~#2\\} % Custom command for adding a shorcut with blanks

% Custom command for alternate subsection titles
    \newcommand{\sectiontitle}[1]{\paragraph{#1} \ \\  \medskip\hrule\medskip} 

%%%%%%%%%%%%%%%%%%%%%%%%%%%%%%%%%%%%%%%%%%%%%%%%%%%%%%%%%%%%%%%%%%%%%%%%%%%%%%%%%%%%%%%%%%%%%%%%

\title{\mytitle}
\begin{document}
\raggedright
\footnotesize

% ------------------------------------------------------------------------------------------------ TITLE
\begin{center}
    \Large{\textbf{\myTitle}} \\
    \Large{\normalsize Version \myVersionnumber} \\
\end{center}
\medskip

% ----------------------------------------------------------------------- start multicol
\begin{multicols}{3}
\setlength{\premulticols}{1pt}
\setlength{\postmulticols}{1pt}
\setlength{\multicolsep}{1pt}
\setlength{\columnsep}{2pt}

These shortcuts and descriptions are inspired by the awesome guide: \textit{"Onramp to Vim"}\cite{onramp}, an austrian talk script \textit{"Vim für Fortgeschrittene"}\cite{vim_advanced} and at least the \textit{"Vim Reference Manual"}\cite{vimref}. Another good ressource for starting with Vim is the online Version of the vimtutor\cite{onlinetut}.
To consolidate and improve your VIM knowledge, VimGolf \cite{vimgolf} is a very good game in the form of a competition.
While this cheat sheet is a selection of commands for one page, there are many more online.\cite{vimcheat}

% ------------------------------------------------------------------------------------------------ MOTIONS
\medskip
\section{Motions (nouns)}
Motions are in the sense of movement.\\
See Vim's help page for motions for a full listing: \texttt{:h motion}

% --------------------------------------------------------------------- Moving within a line
\subsection{Moving within a line}
\command{\texttt{h}, \texttt{l}}{move left/right by character}
\command{\texttt{w}}{move forward one (\textbf{w})ord}
\command{\highLight[Apricot]{\texttt{b}}}{\highLight[Apricot]{move to (\textbf{b})eginning of a word}}
\command{\texttt{e}}{move forward to the (\textbf{e})nd of a word}

% --------------------------------------------------------------------- Jumping within a line
\medskip
\subsection{Jumping within a line}
\command{\texttt{f<char>}}{(\textbf{f})ind a character forward in a line and move to it}
\command{\texttt{t<char>}}{find a character forward in a line and move un(\textbf{t})il it}
\commandbl{}{(one character before)}
\command{\texttt{F<char>}}{(\textbf{f})ind a character backward in a line and move to it}
\command{\texttt{T<char>}}{find a character backward in a line and move un(\textbf{t})il it}
\command{\highLight[Apricot]{\texttt{;}}}{\highLight[Apricot]{repeat last \texttt{f}, \texttt{t}, \texttt{F}, or \texttt{T} command}}
\command{\texttt{,}}{repeat last \texttt{f}, \texttt{t}, \texttt{F}, or \texttt{T} command, but in opposite direction}

% --------------------------------------------------------------------- Moving between lines
\medskip
\subsection{Moving between lines}
\command{\texttt{j}, \texttt{k}}{move up/down one line}
\command{\texttt{H}, \texttt{M}, \texttt{L}}{move (\textbf{H})igh, (\textbf{M})iddle, or (\textbf{L})ow within the viewport}
\command{\highLight[Apricot]{\texttt{n}}}{\highLight[Apricot]{repeat last search - (\textbf{n})ext}}
\command{\texttt{N}}{repeat last search in opposite direction}
\command{\highLight[Apricot]{\texttt{<ctrl>u}, \texttt{<ctrl>d}}}{\highLight[Apricot]{moves (\textbf{u})p or (\textbf{d})own half of a page}}
\command{\texttt{<NN>G}}{(\textbf{G})o to line number \textit{NN}}
\command{\texttt{<NN>gg}}{(\textbf{G})o to line number \textit{NN}}
\command{\highLight[Apricot]{\texttt{\%}}}{\highLight[Apricot]{move cursor to matching character}}
\commandbl{}{(default supported pairs: '()', '\{\}', '[]';}
\commandbl{}{use :h matchpairs in vim for more info)}

% --------------------------------------------------------------------- Inserting text
\medskip
\subsection{Inserting text}
\command{\texttt{a}}{(\textbf{a})ppend text after the cursor}
\command{\texttt{A}}{(\textbf{a})ppend text at the end of the line}
\command{\texttt{i}}{(\textbf{i})nsert text before the cursor}
\command{\texttt{I}}{(\textbf{I})nsert text before the first non-blank in the line}
\command{\texttt{o}}{(\textbf{o})pen a new line below the current line, append text}
\command{\texttt{O}}{(\textbf{O})pen a new line above the current line, append text}

%\columnbreak

% ------------------------------------------------------------------------------------------------ TEXT-OBJECTS
\medskip
\section{Text Objects (nouns)}
Text objects allow you to run the command from anywhere inside the text object\\
For a complete listing, see \texttt{: h text-objects}

\medskip
\textbf{w}ord\\
\command{\texttt{\highLight[Apricot]{a}w}}{\highLight[Apricot]{(\textbf{a})round} (\textbf{w})ord, \highLight[Apricot]{\underline{includes} surrounding white space}}
\command{\texttt{\highLight[Apricot]{i}w}}{\highLight[Apricot]{(\textbf{i})nner} (\textbf{w})ord, does \highLight[Apricot]{\underline{not include} surround white space}}

\medskip
Others: \texttt{[a|i]} +\\
\command{\texttt{\textbf{s}}}{(\textbf{s})entence}
\command{\texttt{\textbf{p}}}{(\textbf{p})arapgraph}
\command{\texttt{\textbf{"}}}{"double quoted string"}
\command{\texttt{\textbf{)}}}{(parenthesized block)}
\command{\texttt{\textbf{t}}}{\texttt{\textbf{\scriptsize<h1>}}single (\textbf{t})ag\texttt{\textbf{\scriptsize</h1>}}}


% ------------------------------------------------------------------------------------------------ COMMANDS
\medskip
\section{Commands (verbs)}
After pressing the key mapping for a given command, Vim will wait for you to identify the text on which the command should operate. The simplest commands are made by simply repeating the operator a second time to act on the current line.\\
For example, where \texttt{d} is the operator for "delete", \texttt{dd} will delete the whole line. Each of \texttt{yy}, \texttt{cc}, \texttt{>\@>}, \texttt{==} behave similarly.

\command{\texttt{d}}{(\textbf{d})elete}
\command{\texttt{c}}{(\textbf{c})hange}
\command{\texttt{y}}{(\textbf{y})ank (copy)}
\command{\texttt{v}}{(\textbf{v})isually select}
\command{\texttt{>}, \texttt{<}}{indent, dedent}
\command{\texttt{=}}{reformat (reindent, break long lines, etc.}


% ------------------------------------------------------------------------------------------------ STRUCTURE
\medskip
\section{The Structure of an Editing Command (sentence)}
\begin{verbatim}
    <number><command><text object or motion>
\end{verbatim}

\subsection{Examples: Command + Motion}
The following table shows some of the many variations of the delete operation that you can build by combining the \texttt{d} operator mapping with a motion:

\medskip
\command{\texttt{dw}}{(\textbf{d})elete from current position to the next (\textbf{w})ord}
\command{\texttt{de}}{(\textbf{d})elete to to the (\textbf{e})nd of the \underline{current} word}
\command{\texttt{d2e}}{(\textbf{d})elete to to the (\textbf{e})nd of the \underline{next} word}
\command{\texttt{dj}}{(\textbf{d})elete down a line (current and one below)}
\command{\texttt{dt)}}{(\textbf{d})elete forward un(\textbf{t})il closing paranthesis}
\command{\texttt{d/world}}{(\textbf{d})elete up until the first search match for \textit{world}}
\command{\texttt{d\$}}{(\textbf{d})elete to (\textbf{\$}) end of line}


% ------------------------------------------------------------------------------------------------ FOO
\medskip
\section{Foo}

\subsection{bar 1}
\command{\texttt{"+y\{motion\}}}{yank into the system clipboard register}
\command{\texttt{"+p}}{paste from the system clipboard register}

%\subsection{bar 2}
\command{\highLight[Apricot]{\texttt{<ctrl>r}}}{\highLight[Apricot]{redo}}
\command{\highLight[Apricot]{\texttt{.}}}{\highLight[Apricot]{repeat last command}}
\command{\highLight[Apricot]{\texttt{<ctrl>6}}}{\highLight[Apricot]{toggle last two Buffers}}

%\subsection{bar 3}
\command{\texttt{:messages}}{startup errors}
\command{\texttt{:changes}}{list of changes}

\subsection{Search and replace}
\command{\texttt{/<pattern>}}{search for \textit{pattern}}
\command{\texttt{?<pattern>}}{search backward for \textit{pattern}}
\command{\highLight[Apricot]{\texttt{n}}}{\highLight[Apricot]{repeat search in same direction}}
\command{\texttt{N}}{repeat search in opposite direction}
\command{\highLight[Apricot]{\texttt{:\%s/<old>/<new>/g}}}{\highLight[Apricot]{replace all \textit{old} with \textit{new} throughout file}}
\command{\texttt{:\%s/<old>/<new>/gc}}{replace all \textit{old} with \textit{new} throughout file}
\commandbl{}{with confirmations}
\command{\texttt{:noh}}{nohighlight, removes the highlighting of search results}

\subsection{(Book-)Marks}
\command{\texttt{m<char>}}{set current position for (\textbf{m})ark \textit{char} \underline{within} the file/buffer}
\command{\highLight[Apricot]{\texttt{m<CHAR>}}}{\highLight[Apricot]{set current position for (\textbf{m})ark \textit{CHAR} \underline{global}}}
\command{\texttt{'<char>}}{jump to position of mark \textit{char} \underline{within} the file/buffer}
\command{\highLight[Apricot]{\texttt{m<CHAR>}}}{\highLight[Apricot]{jump to buffer and position of \underline{global} mark \textit{CHAR}}}
\command{\texttt{:jump}}{list of jumps (:jumps?)}
\command{\texttt{<ctrl>i}}{go to newer position in jump list}
\command{\texttt{<ctrl>o}}{go to older position in jump list}


%%%%%%%%%%%%%%%%%%%%%%%%%%%%%%%%%%%%%%%%%%%%%%%%%%%%%%%%%%%%%%%%%%%%%%%%%%%%%%%%%%%%%%%%%%%%%%%%

% ----------------------------------------------------------------------- Resources
\printbibliography


% ----------------------------------------------------------------------- Endnote
\hrule\medskip
\today ~ \myAuthor, \href{\myGithublink}{\myGithublink}

\end{multicols}
\end{document}